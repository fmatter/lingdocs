\documentclass{article}
\usepackage{geometry}
\usepackage{fancyhdr}

\pagestyle{fancy}
\fancyhf{}
\lhead{\today}
% \lhead{}
\rhead{\footnotesize \itshape Made with \href{https://github.com/fmatter/pylingdocs/}{pylingdocs}}
\cfoot{\thepage}

\input{preamble.tex}
\addbibresource{sources.bib}

\title{Pylingdocs: A Demo}
\author{Florian Matter}
\date{\today\\version 0.0.1.draft}
\newGlossingAbbrev{cor}{coreferential}
\newGlossingAbbrev{hod}{hodiernal past}
\newGlossingAbbrev{cont}{continuous}
\newGlossingAbbrev{part}{particle}
\newGlossingAbbrev{ideo}{ideophone}
\newGlossingAbbrev{iter}{iterative}
\newGlossingAbbrev{rem}{remove past}
\newGlossingAbbrev{uncert}{uncertainty}
\newGlossingAbbrev{inan}{inanimate}
\newGlossingAbbrev{aff}{affirmative}
\newGlossingAbbrev{npst}{non-past}
\newGlossingAbbrev{pert}{pertensive}
\newGlossingAbbrev{ill}{illative}
\newGlossingAbbrev{sup}{supine}
\newGlossingAbbrev{obl}{oblique}
\newGlossingAbbrev{3}{third person}
\newGlossingAbbrev{1}{first person}
\newGlossingAbbrev{poss}{possessive}
\newGlossingAbbrev{dist}{distal}
\newGlossingAbbrev{2}{second person}
\newGlossingAbbrev{loc}{locative}
\newGlossingAbbrev{nmlz}{nominalizer/nominalization}
\newGlossingAbbrev{prox}{proximal/proximate}
\newGlossingAbbrev{voc}{vocative}
\newGlossingAbbrev{rel}{relative}
\newGlossingAbbrev{aux}{auxiliary}
\newGlossingAbbrev{pl}{plural}
\newGlossingAbbrev{ins}{instrumental}
\newGlossingAbbrev{cop}{copula}
\newGlossingAbbrev{acc}{accusative}



\begin{document}
\maketitle

\tableofcontents


\section{\texorpdfstring{Introduction \label{sec:intro}}{Introduction }}

This document does double service as a test for \texttt{pylingdocs} and
a showcase of its capabilities. It aims to demonstrate every feature and
model currently available in \texttt{pylingdocs}.

\section{Common markdown}

You can use all the familiar markdown components. Here is a link to the
\href{https://github.com/fmatter/pylingdocs/}{pylingdocs github repo}.
Here is some \textbf{bold} and \emph{italic} and \textbf{\emph{bold
italic}} text.

\begin{enumerate}
\def\labelenumi{\arabic{enumi}.}
\tightlist
\item
  here
\item
  is
\item
  a
\item
  numbered
\item
  list
\end{enumerate}

and of course here is

\begin{itemize}
\tightlist
\item
  and one
\item
  with
\item
  bullet points\footnote{And here is a (foot)note. You can use markdown
    in here: see \cref{sec:data} for details about Apalaí \obj{-se}
    `\gl{sup}' \parencites[77]{koehn1986apalai}.}
\end{itemize}

A quote:

\begin{quote}
Locating an individual language on a given point of the
ergativity-nominativity axis and the diachronic interpretation of this
axis seem to be conceptually different concerns, even if we were to
assume that there are principies favouring one direction over the other.
\parencites[71]{alvarez1998split}
\end{quote}

\section{\texorpdfstring{Pylingdocs markdown
\label{pld-md}}{Pylingdocs markdown }}

Apart from database references, discussed in \cref{sec:sources}, there
are a number of \texttt{pylingdocs}-specific commands, all patterning
like links:

\begin{itemize}
\tightlist
\item
  cross-references: \cref{common-markdown} or \cref{sec:intro}, see
  corresponding \texttt{label} commands
\item
  example references:

  \begin{itemize}
  \tightlist
  \item
    single \exref[]{ekiri-13}
  \item
    subexample \exref[]{ekiri-10}
  \item
    range: \exref[][ekiri-11]{ekiri-13}
  \item
    or bare: \ref{ekiri-11}
  \end{itemize}
\item
  glosses: \gl{acc}
\item
  todos:
\item
  tables (with automatically generated table labels like
  \cref{tab:consonants}):
\end{itemize}

\begin{table}
\caption{Consonant phonemes of Yawarana}
\label{tab:consonants}
\centering
\begin{tabular}{llllll}
\toprule
          & bilabial & alveolar & palatal & velar & glottal \\
\midrule
occlusive &     /p / &     /t / &    /ch/ &   /k/ &         \\
    nasal &     /m / &     /n / &         &       &         \\
fricative &          &     /s / &         &       &     /j/ \\
   liquid &          &     /r / &         &       &         \\
    glide &     /w / &          &     /y/ &       &         \\
\bottomrule
\end{tabular}

\end{table}

\begin{itemize}
\tightlist
\item
  figures (with automatically generated table labels like
  \cref{fig:cognates}):
\end{itemize}

{[}Cognate identification strategy: cognates.jpg{]}

\section{\texorpdfstring{Other linguistic data
\label{sec:data}}{Other linguistic data }}

\subsection{Native CLDF components}

\begin{itemize}
\tightlist
\item
  forms: Tiriyó \obj{pakoro se wae} `I want a house'
  \parencites[417]{triomeira1999}
\item
  languages: Hixkaryána
\item
  cognate sets:
\end{itemize}

louse-1

\begin{longtable}[]{@{}lllllll@{}}
\toprule()
Form & Language & - & - & - & - & - \\
\midrule()
\endhead
\emph{jamï} & Tiriyó & - & j & a & m & o \\
\emph{azamo} & Apalaí & a & z & a & m & o \\
\emph{əjamo} & Wayana & ə & j & a & m & o \\
\bottomrule()
\end{longtable}

\subsection{Non-native components}

Tiriyó \obj{-e} `\gl{sup}' \parencites[327]{triomeira1999} is a variant
of Tiriyó \obj{-(s)e} \parencites[327]{triomeira1999}. Neither occur on
Tiriyó \obj{mahto} `fire' \parencites[314]{triomeira1999}, because it is
a noun. They are related to Apalaí \obj{-se}
\parencites[77]{koehn1986apalai} and Wayana \obj{-(h)e}
\parencites[236]{wayanatavares2005}. This is thus a cognate set shared
by Apalaí, Tiriyó, and Wayana.

\begin{itemize}
\item
  If Tiriyó \obj{kure} `good / pretty / well'
  \parencites[345]{triomeira1999} has too long a translation, try Tiriyó
  \obj{kure} `good' \parencites[345]{triomeira1999}.
\item
  This dataset contains the Ikpeng text ``The old man''.
\end{itemize}

\section{Interlinear examples}

\ex  Ikpeng  \\\label{ekiri-13}
\begingl \glpreamble nen tan nen ïwïn. //
\gla nen tan nen ɨ-wɨ-n//
\glb \gl{inan}.\gl{prox} here \gl{inan}.\gl{prox} \gl{1}\gl{poss}-machete-\gl{pert}//
\glft ‘“My machete is here.”’ (personal knowledge
)//
\endgl
\xe

\pex\label{}    \a     \label{ekiri-9}        \begingl
        \glpreamble otumunto mun eto ïwïn otumunto //
        \gla otumunto mun eto ɨ-wɨ-n otumunto//
        \glb where \gl{inan}.\gl{dist} \gl{uncert} \gl{1}\gl{poss}-machete-\gl{pert} where//
            \glft ‘“Where might my machete be, where?”’//  
        \endgl 
    \a     \label{ekiri-10}        \begingl
        \glpreamble nento nento yengli ïwïn //
        \gla nen-to nen-to j-eŋ-lɨ ɨ-wɨ-n//
        \glb \gl{inan}.\gl{prox}-\gl{loc} \gl{inan}.\gl{prox}-\gl{loc} \gl{1}>\gl{3}-put-\gl{hod} \gl{1}\gl{poss}-machete-\gl{pert}//
            \glft ‘“Here, here I put my machete.”’//  
        \endgl 
\xe

\ex  Tiriyó  \\\label{tri-1}
\begingl \glpreamble pai iwae teese wïraapa //
\gla pai i-wae t-ee-se wïraapa//
\glb tapir \gl{3}-super \gl{npst}-\gl{cop}-\gl{npst} bow//
\glft ‘The bow was stronger than the tapir.’ (\cite[420]{triomeira1999}
)//
\endgl
\xe

\ex \label{tri-1}
\begingl \glpreamble pai iwae teese wïraapa //
\gla pai i-wae t-ee-se wïraapa//
\glb tapir \gl{3}-super \gl{npst}-\gl{cop}-\gl{npst} bow//
\glft ‘The bow was stronger than the tapir.’ (\cite[420]{triomeira1999}
)//
\endgl
\xe

\ex  Tiriyó  \\\label{tri-1}
\begingl 
\gla pai i-wae t-ee-se wïraapa//
\glb tapir \gl{3}-super \gl{npst}-\gl{cop}-\gl{npst} bow//
\glft ‘The bow was stronger than the tapir.’ (\cite[420]{triomeira1999}
)//
\endgl
\xe

\ex  Tiriyó (\cite[420]{triomeira1999}
) \\\label{tri-1}
\begingl \glpreamble pai iwae teese wïraapa //
\gla pai i-wae t-ee-se wïraapa//
\glb tapir \gl{3}-super \gl{npst}-\gl{cop}-\gl{npst} bow//
\glft ‘The bow was stronger than the tapir.’ (metaphorically)//
\endgl
\xe

\ex  Tiriyó  \\\label{tri-1}
\begingl \glpreamble pai iwae teese wïraapa //
\gla pai i-wae t-ee-se wïraapa//
\glb tapir \gl{3}-super \gl{npst}-\gl{cop}-\gl{npst} bow//
\glft ‘The bow was stronger than the tapir.’ (Sérgio Meira, p.c.
)//
\endgl
\xe

\section{\texorpdfstring{Citing literature
\label{sec:sources}}{Citing literature }}

\begin{itemize}
\tightlist
\item
  see \textcites{alvarez1998split} or
  \textcites[133-134]{alvarez1998split}
\item
  with parentheses:

  \begin{itemize}
  \tightlist
  \item
    ``Locating an individual language on a given point of the
    ergativity-nominativity axis and the diachronic interpretation of
    this axis seem to be conceptually different concerns''
    \parencites{alvarez1998split}
  \item
    ``Locating an individual language on a given point of the
    ergativity-nominativity axis and the diachronic interpretation of
    this axis seem to be conceptually different concerns''
    \parencites[71]{alvarez1998split}
  \end{itemize}
\item
  multiple citations:

  \begin{itemize}
  \tightlist
  \item
    \textcites[133-134]{alvarez1998split}[218]{triomeira1999}
  \item
    \parencites[133-134]{alvarez1998split}[218]{triomeira1999}
  \end{itemize}
\end{itemize}

\printbibliography

\end{document}